\subsection{Unit Type}
\label{sec:unit-type}

\begin{align*}
  \text{Type $\A$, $\B$, $\C$}
    \bnf   {}& \ldots                   && \text{previous types}\\
    \bnfor {}& \Unit                    && \text{unit type} \\
  \\
  \text{Term $\uu$, $\vv$, $\ww$}
    \bnf   {}& \ldots                   && \text{previous terms} \\
    \bnfor {}& \unit                    && \text{the unit inhabitant}
\end{align*}

\newcommand{\rlTyUnit}{\referTo{ty-unit}{rul:ty-unit}}
\newcommand{\showTyUnit}{%
  \infer[\rulename{ty-unit}] % TyUnit
  {\isctx{\G}}
  {\istype{\G}{\Unit}}
}

\newcommand{\rlTermUnit}{\referTo{term-unit}{rul:term-unit}}
\newcommand{\showTermUnit}{%
  \infer[\rulename{term-unit}] % TermUnit
  {\isctx{\G}}
  {\isterm{\G}{\unit}{\Unit}}
}

\newcommand{\rlEqTySubstUnit}{\referTo{eq-ty-subst-unit}{rul:eq-ty-subst-unit}}
\newcommand{\showEqTySubstUnit}{%
  \infer[\rulename{eq-ty-subst-unit}] % EqTySubstUnit
  {\issubst{\sbs}{\G}{\D}}
  {\eqtype{\G}{\subst{\Unit}{\sbs}}{\Unit}}
}

\newcommand{\rlEqSubstUnit}{\referTo{eq-subst-unit}{rul:eq-subst-unit}}
\newcommand{\showEqSubstUnit}{%
  \infer[\rulename{eq-subst-unit}] % EqSubstUnit
  {\issubst{\sbs}{\G}{\D}}
  {\eqterm{\G}{\subst{\unit}{\sbs}}{\unit}{\Unit}}
}

\newcommand{\rlUnitEta}{\referTo{unit-eta}{rul:unit-eta}}
\newcommand{\showUnitEta}{%
  \infer[\rulename{unit-eta}] % UnitEta
  {\isterm{\G}{\uu}{\Unit} \\
   \isterm{\G}{\vv}{\Unit}
 }
 {\eqterm{\G}{\uu}{\vv}{\Unit}}
}


\begin{mathpar}
  {\label{rul:ty-unit} \showTyUnit}

  {\label{rul:term-unit} \showTermUnit}

  {\label{rul:eq-ty-subst-unit} \showEqTySubstUnit}

  {\label{rul:eq-subst-unit} \showEqSubstUnit}

  {\label{rul:unit-eta} \showUnitEta}
\end{mathpar}

Remark: None of the rules can be used to conclude
$\istype{\G}{\Id{\A}{\uu}{\vv}}$ and so lemma~\ref{pbm:id-inversion} still
holds with this extension.

\subsubsection*{Rule {\rlTyUnit}}

Consider a derivation ending with
%
\begin{equation*}
  \showTyUnit
\end{equation*}
%
We have $\isctx{\G}$ as a premise.

\subsubsection*{Rule {\rlTermUnit}}

Consider a derivation ending with
%
\begin{equation*}
  \showTermUnit
\end{equation*}
%
We have $\isctx{\G}$ as a premise and from it and {\rlTyUnit} we conclude
$\istype{\G}{\Unit}$.

\subsubsection*{Rule {\rlEqTySubstUnit}}

Consider a derivation ending with
%
\begin{equation*}
  \showEqTySubstUnit
\end{equation*}
%
By induction hypothesis we have $\isctx{\G}$ and $\isctx{\D}$.
Then with {\rlTyUnit} we get $\istype{\G}{\Unit}$ and $\istype{\D}{\Unit}$.
With {\rlTySubst} we finally conclude $\istype{\G}{\subst{\Unit}{\sbs}}$.

\subsubsection*{Rule {\rlEqSubstUnit}}

Consider a derivation ending with
%
\begin{equation*}
  \showEqSubstUnit
\end{equation*}
%
By induction hypothesis $\isctx{\G}$ and $\isctx{\D}$.
Then using {\rlTyUnit} we conclude $\istype{\G}{\Unit}$.
Applications of {\rlTermUnit} yield $\isterm{\G}{\unit}{\Unit}$
and $\isterm{\D}{\unit}{\Unit}$.
With {\rlTermSubst} we get
$\isterm{\G}{\subst{\unit}{\sbs}}{\subst{\Unit}{\sbs}}$.
Finally using {\rlEqTySubstUnit} and {\rlTermTyConv} we conclude
$\isterm{\G}{\subst{\unit}{\sbs}}{\Unit}$.

\subsubsection*{Rule {\rlUnitEta}}

Consider a derivation ending with
%
\begin{equation*}
  \showUnitEta
\end{equation*}
%
The induction hypothesis on the left premise gives $\isctx{\G}$. By {\rlTyUnit} we then
get $\istype{\G}{\Unit}$, and the rest from the premises.

%%% Local Variables:
%%% mode: latex
%%% TeX-master: "main"
%%% End:
