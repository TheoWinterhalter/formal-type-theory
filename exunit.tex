\subsection{Unit Type}
\label{sec:unit-type}

\begin{align*}
  \text{Type $\A$, $\B$, $\C$}
    \bnf   {}& \ldots                   && \text{previous types}\\
    \bnfor {}& \Unit                    && \text{unit type} \\
  \\
  \text{Term $\uu$, $\vv$, $\ww$}
    \bnf   {}& \ldots                   && \text{previous terms} \\
    \bnfor {}& \unit                    && \text{the unit inhabitant} \\
    \bnfor {}& \ulet{\A}{\uu} \vv       && \text{the unit eliminator}
\end{align*}

\newcommand{\rlTyUnit}{\referTo{ty-unit}{rul:ty-unit}}
\newcommand{\showTyUnit}{%
  \infer[\rulename{ty-unit}] % TyUnit
  {\isctx{\G}}
  {\istype{\G}{\Unit}}
}

\newcommand{\rlTermUnit}{\referTo{term-unit}{rul:term-unit}}
\newcommand{\showTermUnit}{%
  \infer[\rulename{term-unit}] % TermUnit
  {\isctx{\G}}
  {\isterm{\G}{\unit}{\Unit}}
}

\newcommand{\rlTermLet}{\referTo{term-let}{rul:term-let}}
\newcommand{\showTermLet}{%
  \infer[\rulename{term-let}] % TermLet
  {\istype{\ctxextend{\G}{\Unit}}{\A} \\
   \isterm{\G}{\uu}{\Unit} \\
   \isterm{\G}{\vv}{\subst{\A}{\sbzero{\G}{\Unit}{\unit}}}}
  {\isterm{\G}{\ulet{\A}{\uu} \vv}{\subst{\A}{\sbzero{\G}{\Unit}{\uu}}}}
}

\newcommand{\rlEqTySubstUnit}{\referTo{eq-ty-subst-unit}{rul:eq-ty-subst-unit}}
\newcommand{\showEqTySubstUnit}{%
  \infer[\rulename{eq-ty-subst-unit}] % EqTySubstUnit
  {\issubst{\sbs}{\G}{\D}}
  {\eqtype{\G}{\subst{\Unit}{\sbs}}{\Unit}}
}

\newcommand{\rlEqSubstUnit}{\referTo{eq-subst-unit}{rul:eq-subst-unit}}
\newcommand{\showEqSubstUnit}{%
  \infer[\rulename{eq-subst-unit}] % EqSubstUnit
  {\issubst{\sbs}{\G}{\D}}
  {\eqterm{\G}{\subst{\unit}{\sbs}}{\unit}{\Unit}}
}

\newcommand{\rlEqSubstLet}{\referTo{eq-subst-let}{rul:eq-subst-let}}
\newcommand{\showEqSubstLet}{%
  \infer[\rulename{eq-subst-let}] % EqSubstLet
  {\issubst{\sbs}{\G}{\D} \\
   \istype{\ctxextend{\D}{\Unit}}{\A} \\
   \isterm{\D}{\uu}{\Unit} \\
   \isterm{\D}{\vv}{\subst{\A}{\sbzero{\D}{\Unit}{\unit}}}}
  {\eqterm{\G}
    {\subst{(\ulet{\A}{\uu} \vv)}{\sbs}}
    {\ulet
      {\subst{\A}{\sbshift{\G}{\Unit}{\sbs}}}
      {\subst{\uu}{\sbs}}
      \subst{\vv}{\sbs}
    }
    {\subst{\subst{\A}{\sbzero{\D}{\Unit}{\uu}}}}{\sbs}}
}

\newcommand{\rlLetBeta}{\referTo{let-beta}{rul:let-beta}}
\newcommand{\showLetBeta}{%
  \infer[\rulename{let-beta}] % LetBeta
  {\istype{\ctxextend{\G}{\Unit}}{\A} \\
   \isterm{\G}{\vv}{\subst{\A}{\sbzero{\G}{\Unit}{\unit}}}}
  {\eqterm{\G}
    {\ulet{\A}{\unit} \vv}
    {\vv}
    {\subst{\A}{\sbzero{\G}{\Unit}{\unit}}}}
}

\newcommand{\rlCongLet}{\referTo{cong-let}{rul:cong-let}}
\newcommand{\showCongLet}{%
  \infer[\rulename{cong-let}] % CongLet
  {\eqtype{\ctxextend{\G}{\Unit}}{\A}{\B} \\
   \eqterm{\G}{\uu_1}{\uu_2}{\Unit} \\
   \eqterm{\G}{\vv_1}{\vv_2}{\subst{\A}{\sbzero{\G}{\Unit}{\unit}}}}
  {\eqterm{\G}
    {\ulet{\A}{\uu_1} \vv_1}
    {\ulet{\B}{\uu_2} \vv_2}
    {\subst{\A}{\sbzero{\G}{\Unit}{\uu_1}}}}
}

\begin{mathpar}
  {\label{rul:ty-unit} \showTyUnit}

  {\label{rul:term-unit} \showTermUnit}

  {\label{rul:term-let} \showTermLet}

  {\label{rul:eq-ty-subst-unit} \showEqTySubstUnit}

  {\label{rul:eq-subst-unit} \showEqSubstUnit}

  {\label{rul:eq-subst-let} \showEqSubstLet}

  {\label{rul:let-beta} \showLetBeta}

  {\label{rul:cong-let} \showCongLet}
\end{mathpar}

Remark: None of the rules can be used to conclude
$\istype{\G}{\Id{\A}{\uu}{\vv}}$ and so lemma~\ref{pbm:id-inversion} still
holds with this extension.

\subsubsection*{Rule {\rlTyUnit}}

Consider a derivation ending with
%
\begin{equation*}
  \showTyUnit
\end{equation*}
%
We have $\isctx{\G}$ as a premise.

\subsubsection*{Rule {\rlTermUnit}}

Consider a derivation ending with
%
\begin{equation*}
  \showTermUnit
\end{equation*}
%
We have $\isctx{\G}$ as a premise and from it and {\rlTyUnit} we conclude
$\istype{\G}{\Unit}$.

\subsubsection*{Rule {\rlTermLet}}

Consider a derivation ending with
%
\begin{equation*}
  \showTermLet
\end{equation*}
%
We have $\isctx{\G}$ by induction hypothesis on the middle premise.
With an application of {\rlSubstZero} to the middle premise we get
$\issubst{\sbzero{\G}{\Unit}{\uu}}{\G}{\ctxextend{\G}{\Unit}}$.
With {\rlTySubst} we conclude
$\istype{\G}{\subst{\A}{\sbzero{\G}{\Unit}{\uu}}}$.

\subsubsection*{Rule {\rlEqTySubstUnit}}

Consider a derivation ending with
%
\begin{equation*}
  \showEqTySubstUnit
\end{equation*}
%
By induction hypothesis we have $\isctx{\G}$ and $\isctx{\D}$.
Then with {\rlTyUnit} we get $\istype{\G}{\Unit}$ and $\istype{\D}{\Unit}$.
With {\rlTySubst} we finally conclude $\istype{\G}{\subst{\Unit}{\sbs}}$.

\subsubsection*{Rule {\rlEqSubstUnit}}

Consider a derivation ending with
%
\begin{equation*}
  \showEqSubstUnit
\end{equation*}
%
By induction hypothesis $\isctx{\G}$ and $\isctx{\D}$.
Then using {\rlTyUnit} we conclude $\istype{\G}{\Unit}$.
Applications of {\rlTermUnit} yield $\isterm{\G}{\unit}{\Unit}$
and $\isterm{\D}{\unit}{\Unit}$.
With {\rlTermSubst} we get
$\isterm{\G}{\subst{\unit}{\sbs}}{\subst{\Unit}{\sbs}}$.
Finally using {\rlEqTySubstUnit} and {\rlTermTyConv} we conclude
$\isterm{\G}{\subst{\unit}{\sbs}}{\Unit}$.

\subsubsection*{Rule {\rlEqSubstLet}}

Consider a derivation ending with
%
\begin{equation*}
  \showEqSubstLet
\end{equation*}
%
By induction hypothesis on the left premise we have $\isctx{\G}$.
From {\rlSubstZero} we deduce
$\issubst{\sbzero{\D}{\Unit}{\uu}}{\D}{\ctxextend{\D}{\Unit}}$ and thus,
with two applications of {\rlTySubst} we derive
$\istype{\G}{\subst{\subst{\A}{\sbzero{\D}{\Unit}{\uu}}}{\sbs}}$.
An application of {\rlTermLet} followed by {\rlTermSubst} gives us
$\isterm{\G}
  {\subst{(\ulet{\A}{\uu} \vv)}{\sbs}}
  {\subst{\subst{\A}{\sbzero{\D}{\Unit}{\uu}}}{\sbs}}
$.
%
Applications of {\rlTermSubst} yield
$\isterm{\G}{\subst{\uu}{\sbs}}{\subst{\Unit}{\sbs}}$ and
$\isterm{\G}
  {\subst{\vv}{\sbs}}
  {\subst{\subst{\A}{\sbzero{\D}{\Unit}{\unit}}}{\sbs}}
$.
With {\rlEqSubstUnit} and {\rlTermTyConv} we get
$\isterm{\G}{\subst{\uu}{\sbs}}{\Unit}$.
Now, using {\rlEqTyShiftZero}, {\rlEqTySym}
we get to the type
$\subst
  {\subst{\A}{\sbshift{\G}{\Unit}{\sbs}}}
  {\sbzero{\G}{\subst{\Unit}{\sbs}}{\subst{\unit}{\sbs}}}
$.
Then, if we use {\rlEqTyTrans} with a right hand side obtained through
{\rlEqTyCongZero}, {\rlEqSubstUnit}, {\rlEqTySubstUnit}, and finally
{\rlTermTyConv} we get
$\isterm{\G}
  {\subst{\vv}{\sbs}}
  {\subst
    {\subst{\A}{\sbshift{\G}{\Unit}{\sbs}}}
    {\sbzero{\G}{\Unit}{\unit}}
  }
$.
Then, by applying {\rlSubstShift} and {\rlTySubst} and then {\rlEqTySubstUnit}
followed by {\rlEqCtxExtend} we get
$\istype{\ctxextend{\G}{\Unit}}{\subst{\A}{\sbshift{\G}{\Unit}{\sbs}}}$.
Now, using {\rlTermLet} we get
$\isterm{\G}
  {\ulet
    {\subst{\A}{\sbshift{\G}{\Unit}{\sbs}}}
    {\subst{\uu}{\sbs}}
    \subst{\vv}{\sbs}}
  {\subst
    {\subst{\A}{\sbshift{\G}{\Unit}{\sbs}}}
    {\sbzero{\G}{\Unit}{\subst{\uu}{\sbs}}}
  }
$.
By an application of {\rlEqTyCongZero} and {\rlEqTySubstUnit} we equate
$\subst
  {\subst{\A}{\sbshift{\G}{\Unit}{\sbs}}}
  {\sbzero{\G}{\Unit}{\subst{\uu}{\sbs}}}
$ and
$\subst
  {\subst{\A}{\sbshift{\G}{\Unit}{\sbs}}}
  {\sbzero{\G}{\subst{\Unit}{\sbs}}{\subst{\uu}{\sbs}}}
$.
From this, we apply {\rlEqTyTrans} and {\rlEqTyShiftZero} to equate this with
the type
$\subst
  {\subst{\A}{\sbzero{\D}{\Unit}{\uu}}}
  {\sbs}
$.
Now, by applying {\rlTermTyConv} we can finally conclude that
$\isterm{\G}
  {\ulet
    {\subst{\A}{\sbshift{\G}{\Unit}{\sbs}}}
    {\subst{\uu}{\sbs}}
    \subst{\vv}{\sbs}}
  {\subst
    {\subst{\A}{\sbzero{\D}{\Unit}{\uu}}}
    {\sbs}
  }
$.


\subsubsection*{Rule {\rlLetBeta}}

Consider a derivation ending with
%
\begin{equation*}
  \showLetBeta
\end{equation*}
%
By indcution hypothesis on the right premise we obtain $\isctx{\G}$
and $\istype{\G}{\subst{\A}{\sbzero{\G}{\Unit}{\unit}}}$.
We have $\isterm{\G}{\vv}{\subst{\A}{\sbzero{\G}{\Unit}{\unit}}}$ as a
premise.
Now, knowing $\isterm{\G}{\unit}{\Unit}$ we have, by {\rlTermLet},
$\isterm{\G}{\ulet{\A}{\unit} \vv}{\subst{\A}{\sbzero{\G}{\Unit}{\unit}}}$.

\subsubsection*{Rule {\rlCongLet}}

Consider a derivation ending with
%
\begin{equation*}
  \showCongLet
\end{equation*}
%
By induction hypothesis on the middle premise we have $\isctx{\G}$
and $\isterm{\G}{\uu_1}{\Unit}$.
%
By {\rlSubstZero} we deduce
$\issubst{\sbzero{\G}{\Unit}{\uu_1}}{\G}{\ctxextend{\G}{\Unit}}$.
Since by induction hypothesis on the left premise we have
$\istype{\ctxextend{\G}{\Unit}}{\A}$ we conclude
$\istype{\G}{\subst{\A}{\sbzero{\G}{\Unit}{\uu_1}}}$.
%
By induction hypothesis on the different premises we have
$\istype{\ctxextend{\G}{\Unit}}{\A}$,
$\isterm{\G}{\uu_1}{\Unit}$ and
$\isterm{\G}{\vv_1}{\subst{\A}{\sbzero{\G}{\Unit}{\unit}}}$,
and so by {\rlTermLet} we conclude
$\isterm{\G}{\ulet{\A}{\uu_1} \vv_1}{\subst{\A}{\sbzero{\G}{\Unit}{\uu_1}}}$.
%
On the other hand, induction hypothesis, followed by {\rlCongTySubst} and
{\rlTermTyConv} we have
$\istype{\ctxextend{\G}{\Unit}}{\B}$,
$\isterm{\G}{\uu_2}{\Unit}$ and
$\isterm{\G}{\vv_2}{\subst{\B}{\sbzero{\G}{\Unit}{\unit}}}$,
and with an application of {\rlTermLet} so we can derive
$\isterm{\G}{\ulet{\B}{\uu_2} \vv_2}{\subst{\B}{\sbzero{\G}{\Unit}{\uu_2}}}$.
Now thanks to {\rlTermTyConv} and {\rlEqTySym} it suffices to prove
$\eqtype{\G}
  {\subst{\A}{\sbzero{\G}{\Unit}{\uu_1}}}
  {\subst{\B}{\sbzero{\G}{\Unit}{\uu_2}}}
$.
This is achived by {\rlEqTyTrans} using {\rlCongTySubst} on one side and
{\rlEqTyCongZero} on the other.
