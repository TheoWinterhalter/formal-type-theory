
\section{Translation}
\label{sec:translation}

We need the following translations:
%
\begin{align}
  \label{lbl:ctx-ctr}
  \isctx{\G}
  &\quad\leadsto\quad
  \isctx{\G'}
  \\
  \label{lbl:ty-ctr}
  \G', (\istype{\G}{\A})
  &\quad\leadsto\quad
  \istype{\G'}{\A'}
  \\
  \label{lbl:ty-path}
  % the following is needed if we ever need to construct an equivalence given A',B' and A \equiv B
  (\istype{\G'}{\A'}), (\istype{\G}{\A})
  &\quad\leadsto\quad
  (\istype{\G'}{\A''}), (\G' \vdash \A' \simeq \A'')
  \\
  (\G', A'), (\isterm{\G}{\uu}{\A})
  &\quad\leadsto\quad
  \isterm{\G'}{\uu'}{\A'}
  \\
  % the following is needed if we ever need to construct an equivalence given u', v' and u \equiv v
  % (which happens if we need to construct an equivalence given A',B' and A \equiv B)
  \label{lbl:term-path}
  (\isterm{\G'}{\uu'}{\A'}), (\isterm{\G}{\uu}{\A})
  &\quad\leadsto\quad
  (\isterm{\G'}{\uu''}{\A'}), (\G' \vdash \uu' \simeq \uu'' : \A')
  \\
  \G', (\issubst{\sbs}{\G}{\D})
  &\quad\leadsto\quad
  \issubst{\sbs'}{\G'}{\D'}
  \\
  (\issubst{\sbs'}{\G'}{\D'}), (\issubst{\sbs}{\G}{\D})
  &\quad\leadsto\quad
  (\issubst{\sbs''}{\G'}{\D''}), (\sbs' \simeq \sbs'')
  \\
  \label{lbl:eq-ctx-path}
  \G', (\eqctx{\G}{\D})
  &\quad\leadsto\quad
  (\isctx{\D'}), (\G' \simeq \D')
  \\
  \label{lbl:eq-ty-path}
  \G', \B', (\eqtype{\G}{\A}{\B})
  &\quad\leadsto\quad
  (\istype{\G'}{\A'}), (\G' \vdash \A' \simeq \B')
  \\
  \label{lbl:eq-term-path}
  \G', \A', \vv', (\eqterm{\G}{\uu}{\vv}{\A})
  &\quad\leadsto\quad
  (\isterm{\G'}{\A'}{\uu'}), (\G' \vdash \uu' \simeq \vv' : \A')
\end{align}
%
Remarks:
%
\begin{itemize}
\item \eqref{lbl:ty-path} must give the same result as~\eqref{lbl:eq-ty-path} applied to
  the reflexivity case. This is so that we can use the result of~\eqref{lbl:eq-ty-path} as
  if it were obtained through conversion.
\item \eqref{lbl:term-path} must give the same result as~\eqref{lbl:eq-term-path} applied
  to the reflexivity case, for similar reasons.
\item We will need to know that all the equivalences generated from the derivations of a
  given equation are equal (propositionally pointwise). This is where UIP will come into
  play, since such equivalences are compositions of structural acrobatics and transports
  along paths used in $\rl{eq-reflection}$.
\item We also need to know that translations of contexts have the same shape
  as the original context, and the same goes for translation of types up
  to context isomorphisms.
\item In the rules producing isomorphims, we actually assume that the center
  generated is the same as the one that would be generated by the corresponding
  ``simple'' rule.
\end{itemize}


\subsection{Proof of the translation}
\label{sec:proof-tran}

\Case{ctx-empty}
%
The derivation
%
\begin{equation*}
  \infer{ }{\isctx{\ctxempty}}
\end{equation*}
%
is translated into
%
\begin{equation*}
  \infer{ }{\isctx{\ctxempty}}
\end{equation*}


\Case{ctx-extend}

Consider the derivation
%
%
\begin{equation*}
  \infer{
    \inferrule*{\DD}{\isctx{\G}} \\
    \inferrule*{\EE}{\istype{\G}{\A}} \\
    x \not\in \ctxdom{\G}
  }
  {\isctx{(\ctxextend{\G}{\x}{\A})}}
\end{equation*}
%
By~\eqref{lbl:ctx-ctr} on $\derives{\DD}{\isctx{\G}}$, we get
$\derives{\DD'}{\isctx{\G'}}$.
By~\eqref{lbl:ty-ctr} on $\derives{\EE}{\istype{\G}{\A}}$ and $\G'$, we get
$\derives{\EE'}{\istype{\G'}{\A'}}$.
Since $\ctxdom{\G} = \ctxdom{\G'}$, we have $x \notin \ctxdom{\G'}$ and thus
%
\begin{equation*}
  \infer{
    \inferrule*{\DD'}{\isctx{\G'}} \\
    \inferrule*{\EE'}{\istype{\G'}{\A'}} \\
    x \not\in \ctxdom{\G'}
  }
  {\isctx{(\ctxextend{\G'}{\x}{\A'})}}
\end{equation*}


\Case{ty-ctx-conv}

Consider the derivation
%
%
\begin{equation*}
  \infer{
    \inferrule*{\DD}{\istype{\G}{\A}} \\
    \inferrule*{\EE}{\eqctx{\G}{\D}}
  }
  {\istype{\D}{\A}}
\end{equation*}
%
We need to prove~\eqref{lbl:ty-ctr} and~\eqref{lbl:ty-path}.
First assume we have $\D'$ a translation of $\D$.
From $\derives{\EE}{\eqctx{\G}{\D}}$ and $\D'$, we
apply~\eqref{lbl:eq-ctx-path} (\meta{in the symmetric version}) and we deduce
$\G'$ a translation of $\G$ such that $\G' \simeq \D'$ (meaning that in
particular we have some morphism $f$ from $\G'$ to $\D'$).
From $\G'$ and $\derives{\DD}{\istype{\G}{\A}}$, we apply~\eqref{lbl:ty-ctr}
to get $\derives{\DD'}{\istype{\G'}{\A'}}$.
We can finally build the translation:
%
\begin{equation*}
  \infer{
    \inferrule*{\DD'}{\istype{\G'}{\A'}} \\
    \issubst{f}{\D'}{\G'}
  }
  {\istype{\D'}{\subst{\A'}{f}}}
\end{equation*}
%
\meta{We need to specify who is $f$ (and its derivation) and how it applies.}

Now, let's extend our proof to match~\eqref{lbl:ty-path}.
Assume we are given $\istype{\D'}{\A''}$, let's build an equivalence
$\D' \vdash \subst{\A'}{f} \simeq \A''$.
Since $\G' \simeq \D'$, we have an inverse to $f$, say $g$.
For this we deduce $\istype{\G'}{\subst{\A''}{g}}$.
Using this, together with $\istype{\G'}{\A'}$, we can apply~\eqref{lbl:ty-path}
to get some isomorphism $\G' \vdash \A' \simeq \subst{\A''}{g}$.
From this we can build an isomorphism
$\D' \vdash \subst{\A'}{f} \simeq \subst{\subst{\A''}{g}}{f}$.
But as we said, $g$ and $f$ are inverses, so
$\D' \vdash \subst{\subst{\A''}{g}}{f} \simeq \A''$, hence
$\D' \vdash \subst{\A'}{f} \simeq \A''$.

%%% Local Variables:
%%% mode: latex
%%% TeX-master: "main"
%%% End:
