
\section{Translation}
\label{sec:translation}

We need the following translations:
%
\begin{align}
  \isctx{\G}
  &\quad\leadsto\quad
  \isctx{\G'}
  \\
  \G', (\istype{\G}{\A})
  &\quad\leadsto\quad
  \istype{\G'}{\A'}
  \\
  \label{lbl:ty-path}
  % the following is needed if we ever need to construct an equivalence given A',B' and A \equiv B
  (\istype{\G'}{\A'}), (\istype{\G}{\A})
  &\quad\leadsto\quad
  (\istype{\G'}{\A''}), (\G' \vdash \A' \simeq \A'')
  \\
  (\G', A'), (\isterm{\G}{\uu}{\A})
  &\quad\leadsto\quad
  \isterm{\G'}{\uu'}{\A'}
  \\
  % the following is needed if we ever need to construct an equivalence given u', v' and u \equiv v
  % (which happens if we need to construct an equivalence given A',B' and A \equiv B)
  \label{lbl:term-path}
  (\isterm{\G'}{\uu'}{\A'}), (\isterm{\G}{\uu}{\A})
  &\quad\leadsto\quad
  (\isterm{\G'}{\uu''}{\A'}), (\G' \vdash \uu' \simeq \uu'' : \A')
  \\
  \G', (\issubst{\sbs}{\G}{\D})
  &\quad\leadsto\quad
  \issubst{\sbs'}{\G'}{\D'}
  \\
  (\issubst{\sbs'}{\G'}{\D'}), (\issubst{\sbs}{\G}{\D})
  &\quad\leadsto\quad
  (\issubst{\sbs''}{\G'}{\D''}), (\sbs' \simeq \sbs'')
  \\
  \G', (\eqctx{\G}{\D})
  &\quad\leadsto\quad
  (\isctx{\D'}), (\G' \simeq \D')
  \\
  \label{lbl:eq-ty-path}
  \G', \B', (\eqtype{\G}{\A}{\B})
  &\quad\leadsto\quad
  (\istype{\G'}{\A'}), (\G' \vdash \A' \simeq \B')
  \\
  \label{lbl:eq-term-path}
  \G', \A', \vv', (\eqterm{\G}{\uu}{\vv}{\A})
  &\quad\leadsto\quad
  (\isterm{\G'}{\A'}{\uu'}), (\G' \vdash \uu' \simeq \vv' : \A')
\end{align}
%
Remarks:
%
\begin{itemize}
\item \eqref{lbl:ty-path} must give the same result as~\eqref{lbl:eq-ty-path} applied to
  the reflexivity case. This is so that we can use the result of~\eqref{lbl:eq-ty-path} as
  if it were obtain through conversion.
\item \eqref{lbl:term-path} must give the same result as~\eqref{lbl:eq-term-path} applied
  to the reflexivity case, for similar reasons.
\item We will need to know that all the equivalences generated from the derivations of a
  given equation are equal (propositionally pointwise). This is where UIP will come into
  play, since such equivalences are compositions of structural acrobatics and transports
  along paths used in $\rl{eq-reflection}$.
\end{itemize}




%%% Local Variables:
%%% mode: latex
%%% TeX-master: "main"
%%% End:
